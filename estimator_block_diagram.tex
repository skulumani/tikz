\documentclass[varwidth,border=5mm]{standalone}
\usepackage{my_packages}
\usepackage{tikz_packages}
% TikZ styles for drawing
\tikzstyle{block} = [draw,rectangle,thick,minimum height=2em,minimum width=2em]
\tikzstyle{sum} = [draw,circle,inner sep=0mm,minimum size=2mm]
\tikzstyle{connector} = [->,thick]
\tikzstyle{line} = [thick]
\tikzstyle{branch} = [circle,inner sep=0pt,minimum size=1mm,fill=black,draw=black]
\tikzstyle{guide} = []
\tikzstyle{snakeline} = [connector, decorate, decoration={pre length=0.2cm,
                         post length=0.2cm, snake, amplitude=.4mm,
                         segment length=2mm},thick, magenta, ->]

\renewcommand{\vec}[1]{\ensuremath{\boldsymbol{#1}}} % bold vectors
\def \myneq {\skew{-2}\not =} % \neq alone skews the dash
\begin{document}




% The block diagram code is probably more verbose than necessary
\begin{tikzpicture}[auto, node distance=2cm,>=latex']
    \tikzstyle{block} = [draw, fill=blue!20, rectangle, 
    minimum height=3em, minimum width=6em]
    \tikzstyle{sum} = [draw, fill=blue!20, circle, node distance=1cm]
    \tikzstyle{input} = [coordinate]
    \tikzstyle{output} = [coordinate]
    \tikzstyle{pinstyle} = [pin edge={to-,thin,black}]
    \tikzstyle{branch}=[fill,shape=circle,minimum size=3pt,inner sep=0pt]

    % We start by placing the blocks
    \node [input, name=truth] {};
    \node [branch, left of=truth] (stateestnode) {};

    \node [branch, right of=truth] (branch) {};

    \node [coordinate,below of=branch] (spoofnode) {};
    \node [coordinate,below of=spoofnode] (irsnode) {};

    \node [block, right of=branch] (GPS) {GPS};
    \node [block, below of=GPS] (Spoof) {Spoof};
    \node [block, below of=Spoof] (IRS) {IRS};

    \node [block, right of=Spoof, node distance=3cm] (CAFE) {CAFE};
    \node [coordinate,above of=CAFE] (cafenode) {};
    \node [coordinate, below of=CAFE] (irsoutputnode) {};
    \node [coordinate, right of=CAFE] (cafeoutputnode) {};

    \node [block, below of=IRS] (AP) {AP};
    \node [coordinate, below of=irsoutputnode] (apinputnode) {};
    \node [coordinate, right of=apinputnode] (apinputnode2) {};

    \node [block, left of=AP, node distance=3cm] (Plant) {A/C Plant};
    
    \node [block, left of=Plant, node distance=3cm] (Track) {Target Tracking};

    \draw [draw, ->] (stateestnode) -- node {State} (branch) -- (GPS);
    \draw [draw, ->] (branch) -- (spoofnode) -- (Spoof);
    \draw [draw, ->] (spoofnode) -- (irsnode) -- (IRS);

    \draw [draw,->] (GPS.east) -- (cafenode) -- (CAFE.north);
    \draw [draw,->, dashed] (Spoof) -- (CAFE.west);
    \draw [draw,->] (IRS.east) -- (irsoutputnode) -- (CAFE.south);

    \draw [draw,->] (CAFE.east) -- (cafeoutputnode) -- (apinputnode2) -- (AP.east);

    \draw [draw,->] (AP.west) -- (Plant.east);
    \draw [draw,->] (Plant.west) -- (Track.east); 
    \draw [draw,->,dashed] (Track.north) -- node[label={[rotate=90]State Estimate}] {}  (stateestnode);
    % We draw an edge between the controller and system block to 
    % calculate the coordinate u. We need it to place the measurement block. 
    % \draw [->] (controller) -- node[name=u] {$u$} (system);
    % \node [output, right of=system] (output) {};
    % \node [block, below of=u] (measurements) {Measurements};

    % % Once the nodes are placed, connecting them is easy. 
    % \draw [draw,->] (input) -- node {$r$} (sum);
    % \draw [->] (sum) -- node {$e$} (controller);
    % \draw [->] (system) -- node [name=y] {$y$}(output);
    % \draw [->] (y) |- (measurements);
    % \draw [->] (measurements) -| node[pos=0.99] {$-$} 
    %     node [near end] {$y_m$} (sum);
\end{tikzpicture}


\begin{tikzpicture}[>=latex']
\tikzstyle{block} = [draw,fill=blue!20,minimum size=2em]
% diameter of semicircle used to indicate that two lines are not connected
\def\radius{.7mm} 
\tikzstyle{branch}=[fill,shape=circle,minimum size=3pt,inner sep=0pt]

    % Draw blocks, inputs and outputs
    \foreach \y in {1,2,3,4,5} {
        \node at (0,-\y) (input\y) {$i_\y$};
        \node[block] at (2,-\y) (block\y) {$f_\y$};
        \draw[->] (input\y) -- (block\y);
        \draw[->] (block\y.east) -- +(0.5,0);
    }
    \node[block] at (2,-6) (block6) {$f_6$};
    \draw[->] (block6.east) -- +(0.5,0);

    % Calculate branch point coordinate
    \path (input1) -- coordinate (branch) (block1);

    % Define a style for shifting a coordinate upwards
    % Note the curly brackets around the coordinate.
    \tikzstyle{s}=[shift={(0mm,\radius)}]
    % It would be natural to use the yshift or xshift option, but that does
    % not seem to work when shifting coordinates.

    \draw[->] (branch) node[branch] {}{ % draw branch junction
            \foreach \c in {2,3,4,5} {
                % Draw semicircle junction to indicate that the lines are
                % not connected. The intersection between the lines are
                % calculated using the convenient -| syntax. Since we want
                % the semicircle to have its center where the lines intersect,
                % we have to shift the intersection coordinate using the 's'
                % style to account for this.
                [shift only] -- ([s]input\c -| branch) arc(90:-90:\radius)
                % Note the use of the [shift only] option. It is not necessary,
                % but I have used it to ensure that the semicircles have the
                % same size regardless of scaling.
            }
        } |- (block6);
\end{tikzpicture}
\end{document}
